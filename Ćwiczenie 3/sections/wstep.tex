{\section{Spis przyrządów}}

\begin{itemize}
    \item Omomierz AGILENT 34401A,
    \item Omomierz analogowy Axiomet AX-7003,
    \item Rezystory wzorcowe 1$\Omega$, 10$\Omega$, 10k$\Omega$,
    \item Rezystory na płytce 220k$\Omega$, 6.8k$\Omega$, 100$\Omega$,
    \item Rezystor mocy 2.2k$\Omega$,
    \item Zasilacz nieregulowany
\end{itemize}

{\section{Przebieg i cel ćwiczenia}}

Ćwiczenie polegało na pomiarach rezystancji dostępnych na stanowisku rezystorów za pomocą różnych metod i układów pomiarowych takich jak: ukłda dwupunktowy, układ czteropunktowy, układ poprawnego pomiaru napięcia oraz poprawnego pomiaru prądu, a następnmie porównaniu ich z wartościami odczytanymi z rezystorów. W ćwiczeniu mieliśmy także za zadanie sprawdzić oddziaływanie przewodów pomiarowych na dokładność pomiaru. \\
