{\section{Wstęp}}
W 1821 roku Thomas Seeback odkrył zjawisko nazwane potem zjawiskiem termoelektrycznym. Polega ono na przepływie prądu elektrycznego między spojeniami dwóch różnych metali, w przypadku gdy występuje między nimi różnica temperatur.

Na tym zjawisku opiera się działanie termopar, czyli przyrządów służących do pomiaru temperatury. Dzięki nim jesteśmy w stanie przetworzyć wielkość nieelektryczną jaką jest temperatura na napięcie. Jest to szczególnie przydatne w momencie gdy chcemy przesyłać sygnały na duże odległości, przetwarzać lub gromadzić dane o temperaturze badanego obiektu lub sterować różnymi procesami. Z dodatkowych zalet termopary jesteśmy w stanie wymienić jej niezawodność, prostotę użycia, duży zakres pomiarowy oraz mała bezwładność cieplna. \\

\noindent {\textbf{Wzór potrzebny do opracowania pomiarów:}} \\

Temperatura krzepnięcia \dotfill \quad $\displaystyle T_k = \frac{U_k + B}{\alpha}$ \\
{ \small Gdzie:\\
\indent $U_k$ - napięcie krzepnięcia\footnote{Napięcie krzepnięcia wody jesteśmy w stanie uzyskać obliczając średnią arytmetyczną napięć mieszczących się w obszerze plateau.} \\
\indent B - współczynnik równania liniowego \\
\indent $\alpha$ - współczynnik termoelektryczny \\
}
