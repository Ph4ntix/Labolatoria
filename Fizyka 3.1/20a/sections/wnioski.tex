{\section{Wnioski}}

Analiza wyników pomiarów ujawniła znaczną rozbieżność między zmierzoną temperaturą krzepnięcia wody ($-24^{\circ}C$) a wartością oczekiwaną ($0^{\circ}C$), co może potwierdzić błąd złożony, który wysniósł aż $4.68^{\circ}C$. Istotnym źródłem błędu okazało się skalowanie termopary, na które negatywnie wpłynęła ograniczona liczba wykonanych pomiarów oraz problemy z działaniem multimetru. Skutkiem tych czynników było błędne wyznaczenie współczynnika kierunkowego B, który prawdopodobnie został zaniżony o 55 mV.