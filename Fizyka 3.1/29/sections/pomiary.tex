\section{Wyniki i analiza pomiarów}

% ⠀⠀⠀⠀⠀⠀⠀⠀⠀⠀⠀⠀⠀⠀⠀⠀⠀⠀⠀⣀⣤⣤⣤⣶⣤⣤⣀⣀⣀⠀⠀⠀⠀⠀⠀⠀⠀⠀⠀⠀⠀⠀⠀⠀⠀⠀⠀⠀⠀⠀
%⠀⠀⠀⠀⠀⠀⠀⠀⠀⠀⠀⠀⠀⠀⠀⠀⣠⣴⣿⣿⣿⣿⣿⣿⣿⣿⣿⣿⣿⣿⣶⣄⠀⠀⠀⠀⠀⠀⠀⠀⠀⠀⠀⠀⠀⠀⠀⠀⠀⠀
%⠀⠀⠀⠀⠀⠀⠀⠀⠀⠀⠀⠀⠀⠀⢀⣾⣿⣿⣿⣿⣿⡿⠋⠉⠛⠛⠛⠿⣿⠿⠿⢿⣇⠀⠀⠀⠀⠀⠀⠀⠀⠀⠀⠀⠀⠀⠀⠀⠀⠀
%⠀⠀⠀⠀⠀⠀⠀⠀⠀⠀⠀⠀⠀⠀⣾⣿⣿⣿⣿⣿⠟⠀⠀⠀⠀⠀⡀⢀⣽⣷⣆⡀⠙⣧⠀⠀⠀⠀⠀⠀⠀⠀⠀⠀⠀⠀⠀⠀⠀⠀
%⠀⠀⠀⠀⠀⠀⠀⠀⠀⠀⠀⠀⠀⢰⣿⣿⣿⣿⣿⣷⠶⠋⠀⠀⣠⣤⣤⣉⣉⣿⠙⣿⠀⢸⡆⠀⠀⠀⠀⠀⠀⠀⠀⠀⠀⠀⠀⠀⠀⠀
%⠀⠀⠀⠀⠀⠀⠀⠀⠀⠀⠀⠀⠀⢸⣿⣿⣿⣿⣿⠁⠀⠀⠴⡟⣻⣿⣿⣿⣿⣿⣶⣿⣦⡀⣇⠀⠀⠀⠀⠀⠀⠀⠀⠀⠀⠀⠀⠀⠀⠀
%⠀⠀⠀⠀⠀⠀⠀⠀⠀⠀⠀⠀⠀⢨⠟⡿⠻⣿⠃⠀⠀⠀⠻⢿⣿⣿⣿⣿⣿⠏⢹⣿⣿⣿⢿⡇⠀⠀⠀⠀⠀⠀⠀⠀⠀⠀⠀⠀⠀⠀
%⠀⠀⠀⠀⠀⠀⠀⠀⠀⠀⠀⠀⠀⣿⣼⣷⡶⣿⣄⠀⠀⠀⠀⠀⢉⣿⣿⣿⡿⠀⠸⣿⣿⡿⣷⠃⠀⠀⠀⠀⠀⠀⠀⠀⠀⠀⠀⠀⠀⠀
%⠀⠀⠀⠀⠀⠀⠀⠀⠀⠀⠀⠀⠀⢻⡿⣦⢀⣿⣿⣄⡀⣀⣰⠾⠛⣻⣿⣿⣟⣲⡀⢸⡿⡟⠹⡆⠀⠀⠀⠀⠀⠀⠀⠀⠀⠀⠀⠀⠀⠀
%⠀⠀⠀⠀⠀⠀⠀⠀⠀⠀⠀⠀⠀⠀⢰⠞⣾⣿⡛⣿⣿⣿⣿⣰⣾⣿⣿⣿⣿⣿⣿⣿⣿⡇⢰⡇⠀⠀⠀⠀⠀⠀⠀⠀⠀⠀⠀⠀⠀⠀
%⠀⠀⠀⠀⠀⠀⠀⠀⠀⠀⠀⠀⠀⠀⠘⠀⣿⡽⢿⣿⣿⣿⣿⣿⣿⣿⣿⣿⣿⢿⠿⣍⣿⣧⡏⠀⠀⠀⠀⠀⠀⠀⠀⠀⠀⠀⠀⠀⠀⠀
%⠀⠀⠀⠀⠀⠀⠀⠀⠀⠀⠀⠀⠀⠀⠀⠀⣿⣷⣿⣿⣿⣿⣿⣿⣿⣿⣷⣮⣽⣿⣷⣙⣿⡟⠀⠀⠀⠀⠀⠀⠀⠀⠀⠀⠀⠀⠀⠀⠀⠀
%⠀⠀⠀⠀⠀⠀⠀⠀⠀⠀⠀⠀⠀⠀⠀⠀⠙⢿⣿⣿⣿⣿⣿⣿⣿⣿⣿⣿⣿⣿⡟⣹⡿⠇⠀⠀⠀⠀⠀⠀⠀⠀⠀⠀⠀⠀⠀⠀⠀⠀
%⠀⠀⠀⠀⠀⠀⠀⠀⠀⠀⠀⠀⠀⠀⠀⠀⠀⠀⠈⠛⢿⣿⣿⣿⣿⣿⣿⣿⣿⣿⣿⣿⡧⣦⠀⠀⠀⠀⠀⠀⠀⠀⠀⠀⠀⠀⠀⠀⠀⠀
%⠀⠀⠀⠀⠀⠀⠀⠀⠀⠀⠀⠀⢠⡆⠀⠀⠀⠀⠀⠀⠀⠉⠻⣿⣿⣾⣿⣿⣿⣿⣿⣿⡶⠏⠀⠀⠀⠀⠀⠀⠀⠀⠀⠀⠀⠀⠀⠀⠀⠀
%⠀⠀⠀⠀⠀⠀⠀⣀⣠⣤⡴⠞⠛⠀⠀⠀⠀⠀⠀⠀⠀⠀⠀⠚⣿⣿⣿⠿⣿⣿⠿⠟⠁⠀⠀⠀⠀⠀⠀⠀⠀⠀⠀⠀⠀⠀⠀⠀⠀⠀
%⠀⢀⣠⣤⠶⠚⠉⠉⠀⢀⡴⠂⠀⠀⠀⠀⠀⠀⠀⠀⢠⠀⠀⢀⣿⣿⠁⠀⡇⠀⠀⠀⠀⠀⠀⠀⠀⠀⠀⠀⠀⠀⠀⠀⠀⠀⠀⠀⠀⠀
%⠞⠋⠁⠀⠀⠀⠀⣠⣴⡿⠃⠀⠀⠀⠀⠀⠀⠀⠀⠀⣾⠀⠀⣾⣿⠋⠀⢠⡇⠀⠀⠀⠀⠀⠀⠀⠀⠀⠀⠀⠀⠀⠀⠀⠀⠀⠀⠀⠀⠀
%⡀⠀⠀⢀⣷⣶⣿⣿⣿⡇⠀⠀⠀⠀⠀⠀⠀⠀⠀⠀⣿⣆⣼⣿⠁⢠⠃⠈⠓⠦⣄⡀⠀⠀⠀⠀⠀⠀⠀⠀⠀⠀⠀⠀⠀⠀⠀⠀⠀⠀
%⣿⣿⡛⠛⠿⠿⠿⠿⠿⢷⣦⣤⣤⣤⣦⣄⣀⣀⠀⢀⣿⣿⠻⣿⣰⠻⠀⠸⣧⡀⠀⠉⠳⣄⠀⠀⠀⠀⠀⠀⠀⠀⠀⠀⠀⠀⠀⠀⠀⠀
%⠛⢿⣿⣆⠀⠀⠀⠀⠀⠀⠀⠀⠈⠉⠉⠙⠛⠿⣦⣼⡏⢻⣿⣿⠇⠀⠁⠀⠻⣿⠙⣶⣄⠈⠳⣄⡀⠀⠀⠀⠀⠀⠀⠀⠀⠀⠀⠀⠀⠀
%⠀⠀⠈⠋⠀⠀⠀⠀⠀⠀⠀⠀⠀⠀⠁⣐⠀⠀⠀⠈⠳⡘⣿⡟⣀⡠⠿⠶⠒⠟⠓⠀⠹⡄⢴⣬⣍⣑⠢⢤⡀⠀⠀⠀⠀⠀⠀⠀⠀⠀
%⠀⠀⠀⠀⠀⠀⠀⠀⠀⠀⠀⠀⠀⠀⠀⠙⢀⣀⠐⠲⠤⠁⢘⣠⣿⣷⣦⠀⠀⠀⠀⠀⠀⠙⢿⣿⣏⠉⠉⠂⠉⠉⠓⠒⠦⣄⡀⠀⠀⠀
%⠀⠀⠀⠀⠀⠀⠀⠀⠀⠀⠀⠀⠀⠀⠀⠀⠀⠉⠀⠀⠀⠀⠈⣿⣿⣷⣯⠀⠀⠀⠀⠀⠀⠀⠀⠉⠻⢦⣷⡀⠀⠀⠀⠀⠀⠀⠉⠲⣄⠀
%⠠⠀⠀⠀⠀⠀⠀⠀⠀⠀⠀⠀⠀⠀⠀⠀⠀⠀⠀⠀⠘⢦⠀⢹⣿⣏⠀⠀⠀⠀⠀⠀⠀⠀⠀⠀⠀⠀⠙⢻⣷⣄⠀⠀⠀⠀⠀⠀⠈⠳
%⠀⠀⠁⠀⠀⠀⠀⠀⠀⠀⠀⠀⠀⠀⠀⠀⠀⠀⠀⠀⠀⠀⠁⣸⣿⣿⡀⠀⠀⠀⠀⠀⠀⠀⠀⠀⠀⠀⠀⠀⠈⣽⡟⢶⣄⠀⠀⠀⠀⠀
%⠯⠀⠀⠀⠒⠀⠀⠀⠀⠀⠐⠀⠀⠀⠀⠀⠀⠀⠀⠀⠀⠀⠀⢻⣿⣿⣷⣄⠀⠀⠀⠀⠀⠀⠀⠀⠀⠀⠀⠀⠀⢸⣿⡄⠈⠳⠀⠀⠀⠀
%⠀⠀⢀⣀⣀⡀⣼⣤⡟⣬⣿⣷⣤⣀⣄⣀⡀⠀⠀⠀⠀⠀⠀⠈⣿⣿⡄⣉⡀⠀⠀⠀⠀⠀⠀⠀⢀⠀⠀⠀⠀⠀⣿⣿⣄⠀⣀⣀⡀⠀

Pomiary polegały na doprowadzeniu prądu o coraz większej wartości do metalowego drutu
i pomiarze temperatury wraz z zmianą długości drutu.
Dokonaliśmy dwóch serii pomiarów, z czego jedna była z zamontowaną osłoną, a druga bez.
Zebrane dane ukazane są w poniższych tabelach oraz wykresy przyrostu długości do przyrostu temperatury.

\begin{center}
    \begin{tabular}{|c|c||c|c|c|}
        \hline
        \multicolumn{5}{|c|}{Pomiary z osłoną} \\
        \hline
        T     & u(T)    & $\Delta l$ & $\Delta l / l$  & $u_c(\Delta l / l)$ \\ \hline
        $^{\circ}C$ & $^{\circ}C$   & mm      & -          & -          \\ \hline
        24.1  & 0.51205 & 0       & 0          & $1.14 \cdot 10^{-4}$ \\ \hline
        26.3  & 0.51315 & 0.15    & $1.7 \cdot 10^{-4}$ & $1.14 \cdot 10^{-4}$ \\ \hline
        31.7  & 0.51585 & 0.7     & $8 \cdot 10^{-4}$       & $1.14 \cdot 10^{-4}$ \\ \hline
        40    & 0.52    & 1.8     & 0.00206    & $1.14 \cdot 10^{-4}$ \\ \hline
        50.6  & 0.5253  & 3.2     & 0.00366    & $1.14 \cdot 10^{-4}$ \\ \hline
        62.2  & 0.5311  & 4.7     & 0.00537    & $1.14 \cdot 10^{-4}$ \\ \hline
        77.1  & 0.53855 & 6.8     & 0.00777    & $1.14 \cdot 10^{-4}$ \\ \hline
        92.7  & 0.54635 & 9.8     & 0.0112     & $1.14 \cdot 10^{-4}$ \\ \hline
        110.5 & 0.55525 & 11.3    & 0.01291    & $1.14 \cdot 10^{-4}$ \\ \hline
        127   & 0.5635  & 13.7    & 0.01566    & $1.14 \cdot 10^{-4}$ \\ \hline
        147.8 & 0.5739  & 15.9    & 0.01817    & $1.14 \cdot 10^{-4}$ \\ \hline
    \end{tabular}
\end{center}

\begin{center}
    \begin{tabular}{|c|c||c|c|c|}
        \hline
        \multicolumn{5}{|c|}{Pomiary bez osłony} \\
        \hline
        T     & u(T)    & $\Delta l$ & $\Delta l / l$  & $u_c(\Delta l / l)$ \\ \hline
        $^{\circ}C$ & $^{\circ}C$   & mm      & -          & -          \\ \hline
        25.4  & 0.5127  & 0       & 0          & $1.14 \cdot 10^{-4}$ \\ \hline
        26.3  & 0.51315 & 0.1     & $1.14 \cdot 10^{-4}$ & $1.14 \cdot 10^{-4}$ \\ \hline
        29.2  & 0.5146  & 0.5     & $5.72 \cdot 10^{-4}$ & $1.14 \cdot 10^{-4}$ \\ \hline
        34.3  & 0.51715 & 1.2     & 0.00137    & $1.14 \cdot 10^{-4}$ \\ \hline
        41.3  & 0.52065 & 2       & 0.00229    & $1.14 \cdot 10^{-4}$ \\ \hline
        49.3  & 0.52465 & 3.1     & 0.00354    & $1.14 \cdot 10^{-4}$ \\ \hline
        59.6  & 0.5298  & 4.6     & 0.00526    & $1.14 \cdot 10^{-4}$ \\ \hline
        69.5  & 0.53475 & 6       & 0.00686    & $1.14 \cdot 10^{-4}$ \\ \hline
        81.1  & 0.54055 & 7.6     & 0.00869    & $1.14 \cdot 10^{-4}$ \\ \hline
        94.7  & 0.54735 & 9.4     & 0.01074    & $1.14 \cdot 10^{-4}$ \\ \hline
        144   & 0.572   & 15.5    & 0.01771    & $1.14 \cdot 10^{-4}$ \\ \hline
    \end{tabular}
\end{center}

\newpage
\begin{figure}[!ht]
    \centering
    \includegraphics[width = 80mm]{imgs/Graph1.png}
    \includegraphics[width = 80mm]{imgs/Graph2.png}
    \label{fig:wykresy_dlugosci}
\end{figure}

Po wykonaniu regresji liniowej dla powyższych wykresów otrzymaliśmy współczynniki rozszerzalności termicznej. \\

\indent\indent Z osłoną   $\alpha = (1.52227 \pm 0.02865) \cdot 10^{-4} \frac{1}{^{\circ}C}$ \\
\indent\indent Bez osłony $\alpha = (1.51316 \pm 0.01410) \cdot 10^{-4} \frac{1}{^{\circ}C}$ \\

Do obliczenia mocy użyliśmy kolejnych pomiarów dokonywanych jednocześnie z poprzednimi.

\begin{center}
    \begin{tabular}{|c|c||c|c||c|c|}
        \hline
        \multicolumn{6}{|c|}{Pomiary z osłoną} \\
        \hline
        I    & u(I)   & U   & u(U)  & P      & u(P)    \\ \hline
        A    & A      & V   & V     & W      & W       \\ \hline
        0    & 0.01   & 0   & 0.01  & 0      & 0       \\ \hline
        0.19 & 0.0119 & 0.8 & 0.018 & 0.152  & 0.01012 \\ \hline
        0.39 & 0.0139 & 1.5 & 0.025 & 0.585  & 0.02302 \\ \hline
        0.6  & 0.016  & 2.3 & 0.033 & 1.38   & 0.04179 \\ \hline
        0.82 & 0.0182 & 3.1 & 0.041 & 2.542  & 0.06568 \\ \hline
        1.01 & 0.0201 & 3.8 & 0.048 & 3.838  & 0.09047 \\ \hline
        1.23 & 0.0223 & 4.7 & 0.057 & 5.781  & 0.1261  \\ \hline
        1.41 & 0.0241 & 5.4 & 0.064 & 7.614  & 0.15837 \\ \hline
        1.63 & 0.0263 & 6.3 & 0.073 & 10.269 & 0.20399 \\ \hline
        1.82 & 0.0282 & 7   & 0.08  & 12.74  & 0.24529 \\ \hline
        2.02 & 0.0302 & 7.8 & 0.088 & 15.756 & 0.29511 \\ \hline
    \end{tabular}
\end{center}

\begin{center}
    \begin{tabular}{|c|c||c|c||c|c|}
        \hline
        \multicolumn{6}{|c|}{Pomiary bez osłony} \\
        \hline
        I    & u(I)   & U   & u(U)  & P      & u(P)    \\ \hline
        A    & A      & V   & V     & W      & W       \\ \hline
        0    & 0.01   & 0   & 0.01  & 0      & 0       \\ \hline
        0.2  & 0.012  & 0.8 & 0.018 & 0.16   & 0.01025 \\ \hline
        0.4  & 0.014  & 1.5 & 0.025 & 0.6    & 0.02326 \\ \hline
        0.62 & 0.0162 & 2.4 & 0.034 & 1.488  & 0.04423 \\ \hline
        0.8  & 0.018  & 3.1 & 0.041 & 2.48   & 0.06473 \\ \hline
        1.01 & 0.0201 & 3.9 & 0.049 & 3.939  & 0.09271 \\ \hline
        1.21 & 0.0221 & 4.6 & 0.056 & 5.566  & 0.12217 \\ \hline
        1.41 & 0.0241 & 5.4 & 0.064 & 7.614  & 0.15837 \\ \hline
        1.62 & 0.0262 & 6.2 & 0.072 & 10.044 & 0.19998 \\ \hline
        1.82 & 0.0282 & 7   & 0.08  & 12.74  & 0.24529 \\ \hline
        2.4  & 0.034  & 9.3 & 0.103 & 22.32  & 0.40136 \\ \hline
    \end{tabular}
\end{center}

\newpage
\begin{figure}[!ht]
    \centering
    \includegraphics[width = 80mm]{imgs/Graph3.png}
    \includegraphics[width = 80mm]{imgs/Graph4.png}
    \label{fig:wykresy_mocy}
\end{figure}

