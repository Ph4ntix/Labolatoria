\section{Wnioski}

%                     _
%                    | \
%                     '.|
%     _-   _-    _-  _-||    _-    _-  _-   _-    _-    _-
%       _-    _-   - __||___    _-       _-    _-    _-
%    _-   _-    _-  |   _   |       _-   _-    _-
%      _-    _-    /_) (_) (_\        _-    _-       _-
%              _.-'           `-._      ________       _-
%        _..--`                   `-..'       .'
%    _.-'  o/o       :)            o/o`-..__.'        ~  ~
% .-'      o|o                     o|o      `.._.  // ~  ~
% `-._     o|o                     o|o        |||<|||~  ~
%     `-.__o\o                     o|o       .'-'  \\ ~  ~
%          `-.______________________\_...-``'.       ~  ~
%                                    `._______`.
% gul gul gul gul gul gul gul gul

Analizując wyniki pomiarów i patrząc się na współczynnik rozszerzalności cieplnej materiału jesteśmy w stanie wywnioskować, że badany drut jest wykonany z miedzi. Niestety ze względu na małą ilość pomiarów nie byliśmy w stanie uzyskać dokładnej wartości z tabeli która wynosi $16,5^{-4}\frac{1}{^\circ C}$.

Nieliniowość wykresu temperatury od mocy wynika z tego, że opór elektryczny metali wzrasta wraz ze wzrostem jego temperatury, przez co do ogrzewania potrzebna jest coraz większa ilość mocy.