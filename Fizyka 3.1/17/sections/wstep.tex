{\section{Wstęp}}

Przyspieszenie ziemskie, to wielkość wektorowa skierowana do środka Ziemi, która opisuje jak bardzo oddziałuje na ciała znajdujące się w pobliżu siła ciężkości planety.
Wartość przyspieszenia można obliczyć korzystając z prawa powszechnego ciążenia znając masę planety, masę obiektu próbnego, odległość między nimi, oraz stałą grawitacji.
Można skorzystać również z wahadła matematycznego przeprowadzając doświadczenie, przy którym potrzebujemy jedynie długości wahadła, oraz okresu jego oscylacji.
Poza metodą z zastosowaniem wahadła matematycznego zastosowaliśmy też wahadło fizyczne, którym będzie metalowy pierścień. \\

\noindent {\textbf{Wzory potrzebne do wyznaczenia przyspieszenia ziemskiego korzystając z wahadła matematycznego:}} \\

Przyspieszenie ziemskie \dotfill \quad $\displaystyle g = 4\pi^2 \frac{l}{T^2}$ \\
{ \small
    \indent \qquad gdzie $l$ - długość wahadła \\
    \indent \qquad $T$ - okres drgań \\
    \indent \qquad \textit{Wzór ten stosuje się tylko dla małych kątów drgań wahadła! (ok. $8^{\circ}$)} \\
}

Okres drgań \dotfill \quad $\displaystyle  T = \frac{t}{n}$ \\
{ \small
    \indent \qquad gdzie $t$ - czas oscylacji \\
    \indent \qquad $n$ - liczba wachnięć w mierzonym czasie \\
}

\noindent {\textbf{Wzory potrzebne do wyznaczenia przyspieszenia ziemskiego korzystając z wahadła fizycznego:}} \\

Przyspieszenie ziemskie \dotfill \quad $\displaystyle g = 8\pi^2 \frac{I}{T^2 md}$ \\
{ \small
    \indent \qquad gdzie $I$ - moment bezwładności pierścienia względem osi obrotu \\
    \indent \qquad $T$ - okres drgań \\
    \indent \qquad $m$ - masa pierścienia \\
    \indent \qquad $d$ - wewnętrzna średnica pierścienia \\
}

Moment bezwładności $I$ \dotfill \quad $\displaystyle I = I_0 + m \frac{d^2}{4}$ \\
{ \small
    \indent \qquad gdzie $I_0$ - moment bezwładności pierścienia względem osi przechodzącej przez środek masy \\
    \indent \qquad $d$ - wewnętrzna średnica pierścienia \\
    \indent \qquad $m$ - masa pierścienia \\
}

Moment bezwładności $I_0$ \dotfill \quad $\displaystyle I_0 = \frac{1}{8} m(d^2 + D^2)$ \\
{ \small
    \indent \qquad gdzie $d$ - wewnętrzna średnica pierścienia \\
    \indent \qquad $D$ - zewnętrzna średnica pierścienia \\
    \indent \qquad $m$ - masa pierścienia \\
}
