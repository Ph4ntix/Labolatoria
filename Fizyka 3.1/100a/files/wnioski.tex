{\section{Wnioski}}

Ćwiczenie polegało na wyznaczeniu gęstości detali za pomocą suwmiarki i/lub mikrometru.
Gęstości, które udało się wyznaczyć: \\

{\indent $\rho_c = (2.7 \pm 0.95) \cdot 10^{-3} \frac{g}{mm^3}$} \\

{\indent $\rho_w = (3.170 \pm 0.013) \cdot 10^{-3} \frac{g}{mm^3}$} \\

Dzięki temu doświadczeniu udało nam się zrozumieć zasadę działania niepewności pomiarowych w praktyce,
jak również wyznaczyć wartość wielkości złożonej. \\

Z obliczeń wynika, że gęstości obu detali są nieznacznie różne.
Może być to spowodowane nieprzewidywalnymi błędami pomiarowymi,
ponieważ z obliczeń niepewności systematycznych i niepewności standardowej wynika, że odchyłka od wartości wielkości mierzonej prawie dla każdego pomiaru jest sobie równa,
a oba detale wytworzone zostały z tego samego materiału.

