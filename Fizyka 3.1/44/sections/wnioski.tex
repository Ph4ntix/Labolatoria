\newpage
\section{Wnioski}
Ćwiczenie polegało na zaobserwowaniu zmian w rezystancji badanych próbek pod względem temperatury. Ćwiczenie zostało wykonane poprawnie. Analizując zdobyte dane pomiarowe i obliczając współczynnik oporu dla metalu oraz szerokość pasma wzbronionego dochodzimy do wniosków, że wygenerowane wykresy dla rezystancji od temperatury są poprawne i zgodne z naszymi przewidywaniami. Dodatkowo z danych podanych w tabeli jesteśmy w stanie wywnioskować, że trzy z czterech badanych próbek są połprzewodnikami.

Niestety przez zbyt szybkie zmiany temperatury na początku doświadczenia nie byliśmy w stanie dokładnie odczytać wartości rezystancji z miernika przez co udało nam się uzyskać tylko 98\% dopasowania prostej do danych.
