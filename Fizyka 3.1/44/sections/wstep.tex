\section{Wstęp}
Metale oraz półprzewodniki mają zdolność do przewodzenia prądu elektrycznego. Ich zdolność do przewodnictwa jest ściśle związana z temperaturą oraz z właściwościami tego materiału. Nośnikami na które wpływ ma temperatura są elektrony swobodne które poruszają się po całej sieci krystalicznej.

Metale są zbudowane z jonów ułożonych w sieć krystaliczną  oraz wolnych elektronów. Ruch tych elektronów jest swobodny. Atomy sieci krystalicznej metalu cały czas się ruszają w nieuporządkowany sposób. W metalach podczas wzrostu temperatury atomy szybciej drgają utrudniając przemieszczanie się elektronów, czyli wzrasata rezystancja.

Półprzewodniki możemy podzielić na dwie grupy: samoistne oraz domieszkowane. Półprzewodniki samoistne to takie, w których sieć krystaliczna nie jest zaburzona atomami innego pierwiastka. W tym typie półprzewodników koncentracja elektronów silnie zależy od temperatury. W półprzewodnikach możemy zaobserwować zjawisko przeciwne do tego w metalach, czyli im wyższa temperatura tym więcej energii zostało dostarczone do elektronów i zwiększa się przewodnictwo półprzewodnika.
