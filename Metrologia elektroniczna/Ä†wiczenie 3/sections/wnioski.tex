{\section{Wnioski}}
Analizując wyniki pomiarów od początku jesteśmy w stanie zauważyć, że wartości zmierzone w Tabeli 1 za pomocą miernika cyfrowego nieznacznie różnią się od wartości odczytanych. Jest to spowodowane błędem metody pomiaru dwupunktowego. Następnie zauważamy, że nie powinniśmy używać omomierza analogowego do pomiarów rezystancji przez jego niedokładność oraz problemy z dokładnym odczytaniem mierzonej wartości.

W kolejnym podpunkcie wykonaliśmy pomiar rezystancji w układzie czteropunktowym. Zauważamy tutaj znaczącą poprawę dokładności.

Następnie przy wykorzystaniu przewodów różnego typu jesteśmy w stanie zauważyć wpływ rodzaju przewodu na wynik pomiaru. Są to wartości poniżej jednego Ohma, więc jest to problem tylko dla małych rezystancji.

Następnie po dokonaniu pomiarów metodami poprawnego pomiaru napięcia i poprawnego pomiaru prądu zauważyliśmy duże odchyłki dla poszczególnych rezystorów.
Po wyznaczeniu rezystancji granicznej jesteśmy w stanie zauważyć, że wyniki naszych pomiarów potwierdzają zasadę wyboru metody pomiaru. Pomiar poprawnym pomiarem napięcia wykazywał mały błąd dla niskich rezystancji, natomiast poprawny pomiar prądu pozwalał z mniejszym błędem zmierzyć wartości o wysokiej oporności. Jesteśmy w stanie dobrze to zauważyć na przykładzie rezystora $220k\Omega$ oraz $2,2\Omega$.
